\documentclass[12pt,a4paper,reqno]{amsart}
\usepackage[utf8]{inputenc}  % Unixin merkistö
\usepackage[T1]{fontenc}       % kirjaimet, joissa aksentteja (skandit)
\usepackage[finnish]{babel}    % suomalaiset babel-makrot
\usepackage{amsfonts}          % AMS-paketteja
\usepackage{amsmath}
\usepackage{amssymb}
\newcommand{\R}{\mathbb{R}}
\newcommand{\C}{\mathbb{C}}
\newcommand{\Q}{\mathbb{Q}}
\newcommand{\N}{\mathbb{N}}
\newcommand{\Z}{\mathbb{Z}}
\newtheorem{maar}{Määritelmä}[section]
\newtheorem{huom}[maar]{Huomautus}
\newtheorem{lause}[maar]{Lause}
\newtheorem{lemma}[maar]{Lemma}
\newtheorem{teoreema}[maar]{Teoreema}
\newtheorem{esim}[maar]{Esimerkki}
\newtheorem{korollaari}[maar]{Korollaari}
\pagestyle{plain}
 
\title{Kuratowskin upotuslause\\Kandintyö}
\author{Pekka Keipi\\013592712} 
 
\begin{document}
 
\maketitle
\newpage
\section{Funktion rajoittuma, normi ja sup-normi}
\begin{maar}
%\emph{Vektoriavaruus}
Vektoriavaruus\\
\emph{
Joukko $V$ on $\R$-kertoiminen vektoriavaruus, jos kaikkiin $v,w\in V$ ja $a\in \R$ on liitetty yksikäsitteinen summa $v+w\in V$ ja tulo $av\in V$ niin, että seuraavat ominaisuudet ovat voimassa:
\begin{itemize}
\item[i)\phantom{iiv}] $(u+v)+w=u+(v+w)\text{ kaikilla }u,v,w\in V.$
\item[ii)\phantom{iv}] $v+w=w+v\text{ kaikilla }v,w\in V.$
\item[iii)\phantom{v}] On olemassa sellainen $0=0_{v},$ että $v+0=v\text{ kaikilla }v\in V.$
\item[iv)\phantom{ii}] Jokaiseen $v\in V $ liittyy sellainen $-v\in V$, että $v+(-v)=0$.
\item[v)\phantom{iii}] $a(v+w)=av+aw\text{ kaikilla }a\in\R, v,w\in V.$
\item[vi)\phantom{ii}] $(a+b)v=av+bv\text{ kaikilla }a,b\in\R, v\in V.$
\item[vii)\phantom{i}] $a(bv)=(ab)v\text{ kaikilla }a,b\in\R, v\in V.$
\item[viii)] $1_{v}=v$ kaikilla $v\in V.$
\end{itemize}
Tässä avaruuden $V$ alkioita kutsutaan vektoreiksi ja avaruuden $\R$ alkioita skalaareiksi.
}
\end{maar}
%\begin{esim}Tärkein esimerkki vektoriavaruuksista on euklidinen avaruus $\R^n$, jossa luku $n\in \N$ on avaruuden ulottuvuus.\end{esim}
\begin{maar}
Vektorialiavaruus\\
\emph{
Osajoukko $W\in V$ on vektoriavaruuden $V$ (vektori)aliavaruus, jos
\begin{itemize}
\item[i)\phantom{iiv}] $v+w\in W\text{ kaikilla }v,w\in W,$
\item[ii)\phantom{iv}] $av\in W\text{ kaikilla }a\in\R, v\in W$ ja
\item[iii)\phantom{v}] $0_{v}\in W$
\end{itemize}
}
\end{maar}
\begin{maar}
Rajoitettu funktio\\
\emph{
Olkoon $D$ avaruus ja $F(D,\R)$ kaikkien avaruudessa $D$ määriteltyjen reaaliarvoisten funktioiden $f\colon D\rightarrow \R$ joukko. Funktio $f\colon D\rightarrow \R$ on rajoitettu, jos on olemassa sellainen $M\in\R,M\geq 0$, jolla $|f(x)|\leq M$ kaikilla $x\in D$.\\\\
%Rajoitettujen funktioiden avaruutta $\cup f, f\colon  $
Rajoitettujen funktioiden avaruus $Raj(D,\R )$ koostuu kaikista rajoitetuista funktioista $f\colon D\rightarrow \R $.
}
\end{maar}
\begin{lemma}
%Olkoon $D$ avaruus ja $F(D,\R)$ avaruuden $D$ reaaliarvoisten funktioiden avaruus. Tällöin r
Rajoitettujen funktioiden avaruus $Raj(D,\R )$ on reaaliarvoisten funktioiden joukon $F(D,\R)$ aliavaruus.
\end{lemma}
%\noindent
\emph{Todistus.} Olkoon $f,g\in Raj(D,\R),M,N\in\R, M,N\geq 0$ niin, että $|f(x)|\leq M$ ja $|g(x)|\leq N$ kaikilla $x\in D$. Tällöin seuraavat kohdat pätevät: 
\begin{itemize}
\item[i)\phantom{iiv}] $|f(x)|+|g(x)|\leq M+N\leq\infty$, siis $f+g\in Raj(D,\R)$ kaikilla $f,g\in Raj(D,\R),$
\item[ii)\phantom{iv}] $|af(x)|\leq a\cdot M\leq\infty$, siis $af\in Raj(D,\R)$ kaikilla $a\in\R, f\in Raj(D,\R)$ ja
\item[iii)\phantom{v}] $0_{F(D,\R)}(x)=0$ kaikilla $x\in D$, jolloin $0_{F(D,\R)} \in Raj(D,\R )$.
\end{itemize}

\begin{maar} Normi\\
\emph{
Olkoon $E$ vektoriavaruus ja $|\cdot|\colon E\rightarrow \R_+$, $x\mapsto|x|$ kuvaus joukossa $E$. %Kuvaus $|\cdot|$ liittää jokaiseen joukon $E$ alkioon $x\in E$ reaaliluvun $|x|\geq 0$.
Kuvaus $|\cdot|$ on normi avaruudessa $E$, jos seuraavat ominaisuudet pätevät kaikilla $ x,y\in E, a\in\R$.
\begin{itemize}
\item[(N1)]$|x+y|\leq |x|+|y|$,
\item[(N2)]$|ax|=|a||x|$,
\item[(N3)]Jos $|x|=0$, niin $x=\bar{0}$.
\end{itemize} 
Vektoriavaruutta, jossa on annettu jokin normi, sanotaan normiavaruudeksi.
}
\end{maar} 

\begin{esim}
\emph{
Olkoon $\R ^n$ joukko, jossa määritellään tavallinen euklidinen normi $|x| = \sqrt{x^{2}_{1} + \cdots +x^{2}_{n}}$.
}
\end{esim}

\begin{maar} Sup-normi\\
\emph{
Olkoon $D$ epätyhjä avaruus ja $Raj(D,\R )$ kaikkien avaruudessa $D$ määriteltyjen rajoitettujen funktioiden $f\colon D\rightarrow \R$ vektoriavaruus. Yhtälö $||f||=\sup\{|f(x)|\colon x\in D\}$ määrittelee normin avaruudessa $Raj(D,\R )$. %Normia $||\cdot||$ 
Tätä normia sanotaan avaruuden $Raj(D,\R )$ sup-normiksi.
}
\end{maar} 
\emph{Todistus.} 
\begin{itemize}
\item[(N1)] Olkoon $f,g\in Raj(D,\R )$ ja $x\in D$. Tällöin
\begin{equation*}
|f+g|(x)=|f(x)+g(x)|\leq|f(x)+g(x)|\leq||f||+||g||
\end{equation*}
kaikilla $x\in D$, joten $||f+g||\leq||f||+||g||$.

\item[(N2)] Olkoon $f\in Raj(D,\R )$ ja $x\in D$. Tällöin
\begin{equation*}
%\label{N2todistus1}
|af|(x)=|af(x)|=|a||f(x)|\leq|a|||f||
\end{equation*}
kaikilla $x\in D$, joten $||af||\leq|a|||f||$. Jos $a=0$, niin (N2) pätee muodossa $0=0$. %$$||af||=||0\cdot f||=0=|0|||f||=|a|||f||.$$ 
Jos $a\neq 0$, niin $f=a^{-1}af$ ja edellisen
%yhtälön $\ref{N2todistus1}$ 
mukaan $||f||\leq|a^{-1}|||af||$
%$$|f|(x)=|a^{-1}af|(x)=|a^{-1}af(x)|=|a^{-1}||af(x)|\leq|a^{-1}|||af(x)||.$$
ja edelleen $||af||\geq|a|||f||$ kaikilla $x\in D$. Tällöin siis $||af||=|a|||f||$.

\item[(N3)]Jos $||f||=0$, niin $|f(x)|=0$ kaikilla $x\in D$, eli  $f=\bar{0}$.
\end{itemize} 

\begin{esim}
\emph{
Tapauksessa $D=\N$ saadaan kaikkien rajoitettujen jonojen joukko $raj(\N,\R).$ Joukon alkioita ovat rajoitetut jonot $x=(x_1,x_2,\dots).$ Tälle joukolle käytetään usein merkintää $l_\infty$.
}
\end{esim}


\newpage
\section{Metrinen avaruus}
\begin{maar} Metrinen avaruus \emph{on pari $(X,d)$, jossa $X$ on joukko ja $d$ on \emph{metriikka} joukossa $X$.
Tällöin $d\colon X\times X\rightarrow \R_+$ on kuvaus, jolle pätee seuraavat ominaisuudet kaikilla $x,y,z\in X$:
\begin{itemize}
\item[(M1)]$d(x,z)\leq d(x,y)+d(y,z)$
\item[(M2)]$d(x,y)=d(y,x)$
\item[(M3)]$d(x,y)=0$, jos ja vain jos $x=y$.
\end{itemize} }
\end{maar} 

\begin{esim}\emph{
Euklidinen metriikka avaruuden $\R^n$ kahden pisteen $p=(p_1,p_2,\dots,p_n)$ ja $q=(q_1,q_2,\dots,q_n)$ välillä on määritelty $d(p,q)=\sqrt{(p_1-q_1)^2+(p_2-q_2)^2+\dots+(p_n-q_n)^2}$.}
\end{esim}

\begin{esim}\emph{
Normiavaruus on aina metrinen avaruus. 
}\\
Todistus.
\emph{Olkoon $E$ normiavaruus ja $x,y\in E$. 
Metriikaksi voidaan valita $d(x,y)=|x-y|$, jolloin kaikilla $x,y\in E$ pätee
\begin{itemize}
\item[(M1)]%\begin{align*}
$(x,y)=|x-y|=|x+(-y)|\leq|x|+|(-y)|=|x+0|+|0+y|$\\
%\phantom{(x,y)}
$= d(x,0)+d(0,y)$
%(x,y)=&|x-y|=|x+(-y)|\leq|x|+|(-y)|=|x+0|+|0+y|\\=& d(x,0)+d(0,y)
%\end{align*}
\item[(M2)]$d(x,y)=|x-y|=|y-x|=d(y,x)$
\item[(M3)]$d(x,y)=|x-y|=0$, jos ja vain jos $x=y$.
\end{itemize}
}
\end{esim}
\begin{maar}Joukkojen välinen etäisyys\\
\emph{Olkoon $(X,d)$ metrinen avaruus ja $A,B\subset X$. Tällöin etäisyys osajoukkojen $A$ ja $B$ välillä on määritelty $d(A,B)=\inf\{d(a,b)\colon a\in A, b\in B\}$, eli etäisyyksien $d(a,b)$ suurin alaraja, kun $a\in A$ ja $b\in B$. }\\
\\
\emph{Tällöin pätee $d(A,B)\geq 0$ ja jos $A= \emptyset$ tai $B= \emptyset$, niin $d(A,B)= 0$. 
}
\end{maar}
\begin{lause}
\emph{Olkoon $(X,d)$ metrinen avaruus, $x,y\in X$ ja $A\subset X$ epätyhjä. Tällöin $|d(x,A)-d(y,A)|\leq d(x,y)$. Erityisesti $|d(x,z)-d(y,z)|\leq d(x,y)$ kaikilla $z\in X$}\\
\\
Todistus. %\emph{Aputulos: $b+\inf\{a\in A\}=\inf\{b+a\colon a\in A\}$}\\
\emph{Olkoon $a\in A$. Tällöin kaikilla $y\in A$ pätee $d(x,A)\leq d(x,a)\leq d(x,y)+d(y,a)$. Ottamalla infimum kaikkien $a\in A$ yli saadaan $d(x,A)\leq d(x,y)+d(y,A)$
%\begin{equation*}\begin{split} d(x,A)=&\inf\{d(x,b)\colon b\in A\}\leq\inf\{d(x,y)+d(y,b)\colon b\in A\}\\=&d(x,y)+\inf\{d(y,b)\colon b\in A\}=d(x,y)+d(y,A)\end{split}\end{equation*}
ja edelleen $d(x,A)-d(y,A)\leq d(x,y) $.
Vastaavasti olkoon $y\in A$, jolloin kaikilla $x\in X$ pätee $d(y,A)-d(x,A)\leq d(y,x)=d(x,y) $. Nyt $d(x,A)-d(y,A)\leq d(x,y) $ ja $d(y,A)-d(x,A)\leq d(x,y) $, jolloin siis $|d(x,A)-d(y,A)|\leq d(x,y) $.
}
\end{lause}

\begin{maar} Homeomorfismi \emph{tarkoittaa kuvausta $f\colon X\rightarrow Y$, jolla 
\begin{itemize}
\item[(1)] $f$ on bijektio,
\item[(2)] $f$ on jatkuva,
\item[(3)] $f^{-1} \colon Y\rightarrow X$ on jatkuva. 
\end{itemize}
}
\end{maar}

\begin{maar}Upotus \emph{tarkoittaa kuvausta $f\colon X\rightarrow Y$, joka määrittelee homeomorfismin $f_1\colon X\rightarrow f[X]$, jolla $f_1(x)=f(x)$ kaikilla $x\in X$. %Tällöin funktiolla $f$ pätee seuraavat ehdot:
%\begin{itemize}
%\item[(1)] $f$ on injektio,
%\item[(2)] $f$ on jatkuva,
%\item[(3)] $f^{-1} \colon f[X]\rightarrow X$ on jatkuva. 
%\end{itemize}
}
\end{maar}
\begin{maar}Isometria \emph{on etäisyydet säilyttävä kuvaus. Olkoon $(X,d)$ ja $(X',d')$ metrisiä avaruuksia. %Avaruudet $(X,d)$ ja $(X',d')$ ovat isometrisiä,
Tällöin kuvaus $h\colon X\rightarrow X'$ on isometria, jos ja vain jos $d(x_1,x_2)=d'(h(x_1),h(x_2))$ kaikilla $x_1,x_2 \in X$.
}
\end{maar}
%\begin{lemma}
%Isometria on aina myös upotus. \\
%\\
%Todistus.\emph{ Olkoon $(X,d)$ ja $(X',d')$ metrisiä avaruuksia ja kuvaus $h\colon X\rightarrow X'$ isometria. Tällöin $d(x_1,x_2)=d'(h(x_1),h(x_2))$ kaikilla $x_1,x_2 \in X$.
%\begin{itemize}
%\item[(1)] $h$ on bijektio,
%\item[(2)] $h$ on jatkuva,
%\item[(3)] $h^{-1} \colon X'\rightarrow X$ on jatkuva. 
%\end{itemize}
%}
%\end{lemma}
\begin{esim}\emph{
Peilaus, rotaatio ja siirtokuvaus ovat geometriasta tuttuja isometrioita avaruudessa $\R^2$.}
\end{esim}
\newpage
\section{Konveksi verho}
\begin{maar}%%Konveksi joukko sisältää kaikkien alkioidensa väliset janapolut. 
Konveksius. \emph{Olkoon $E$ normiavaruus, $A\subset E$ ja $a,b\in A$. Yhtälö 
$$\alpha (t)=a+t(b-a)=(1-t)a+tb$$ 
määrittelee janapolun $\alpha \colon [0,1]\rightarrow E$. Kuvajoukko $\alpha [0,1] = [a,b]$ on jana, jonka päätepisteet ovat $\alpha (0)=a\in A$ ja $\alpha (1)=b\in A$. 
% Merkitään $\alpha I = [a,b]$. 
Joukko $A\subset E$ on konveksi, jos ja vain jos $[a,b]\subset A$ kaikilla $a,b\in A$ 
%Jos $[a,b]\in A$, niin joukko $A$ on konveksi.%aina kun $a,b\in A$.
}
\end{maar}
\begin{esim}
\emph{
%Avaruus $\R ^n$ on konveksi, sillä minkä tahansa kahden pisteen $a,b\in \R ^n $ välinen jana $[a,b]$ kuuluu avaruuteen $\R ^n$.
Avaruuden $\R ^n$ suljettu yksikkökuula $ \bar B(0,1)= \{x\in \R ^n \colon |x|\leq 1\}$ on konveksi.
}
\end{esim}
\begin{maar}Konveksi verho. \emph{Olkoon $E$ normiavaruus, $A\subset E$ ja $(A_j)_{j\in J}\subset \mathcal{P}(E)$ 
kaikkien sellaisten konveksien osajoukkojen $A_k\subset E$, $k\in J$ muodostama perhe, joilla %joukko $A_k$ konveksi ja 
$A\subset A_k$% kaikilla $k\in J$
.\\ Tällöin joukko $C(A)=\bigcap_{j\in J} A_j$ on joukon $A$ konveksi verho. }
\end{maar}
\begin{lemma} \emph{Konveksi verho $C(A)$ on konveksi}.\end{lemma} 
\emph{Todistus.} Olkoon $a,b\in C(A)=\bigcap_{j\in J} A_j$. 
Tällöin $a,b\in A_i$ kaikilla $i\in J$. %ja koska erityisesti 
Jokainen $A_i$ on konveksi, joten kaikilla $i\in J$ pätee $[a,b]\subset A_i$.
Tällöin kaikilla $a,b\in C(A)$ pätee $[a,b] \subset C(A)$, joten konveksi verho $C(A)$ on konveksi.
%%Tästä seuraa, että konveksi verho $C(A)$ on konveksi. \end{korollaari}
\begin{korollaari}\emph{Konveksi verho $C(A)$ on pienin konveksi joukko, joka sisältää joukon $A$.}
\end{korollaari}
%%Tällöin koska $C(A)=\bigcap_{j\in J} A_j$, niin $C(A)\subset A_i$ kaikilla $i\in J$. Tällöin konveksi verho $C(A)$ on pienin konveksi joukko, joka sisältää joukon $A$.
\emph{Todistus.} Olkoon $B$ sellainen konveksi joukko, jolla $A\subset B\subset C(A)$. 
Joukko $B$ on konveksi ja $A\subset B$, joten $B\in (A_j)_{j\in J}$. 
Tällöin\\ $C(A)=\bigcap_{j\in J} A_j\subset B$%, sillä $C(A)\subset A_i$ kaikilla $i\in J$
. Siis $B= C(A)$ ja $C(A)$ on pienin konveksi joukko, jolla $A\subset C(A)$

\begin{lemma} \emph{Olkoon $E$ normiavaruus ja $A\subset E$. Tällöin joukon $A$ konveksi verho $C(A)$ sisältää kaikki lineaarikombinaatiot joukon $A$ vektoreista. }
\end{lemma}

\emph{Todistus.} 
Olkoon $x_0,x_1,\cdots, x_n, x_{n+1} \in A$, $a,\lambda_0,\lambda_1,\cdots, \lambda_n, \lambda_{n+1} \in \R$ ja $a,\lambda_i\geq 0$ kaikilla $ i\in\N$. Konveksiuden nojalla 
%$a x_1+b x_2 \in C(A)$, kun $a +b =1$. 
$\lambda_0 x_0+\lambda_1 x_1 \in C(A)$, kun $\lambda_0 +\lambda_1=1$.
%Edelleen $\lambda_1 (ax_1+b x_2)+\lambda_2 x_3 \in C(A)$, kun $\lambda_1 +\lambda_2 =1$ ja $a+b=1$. 
%ja edelleen $a (\lambda_0 x_0+\lambda_1 x_1)+\lambda_2 x_2=a\lambda_0 x_0+a\lambda_1  x_1+\lambda_2 x_2 \in C(A)$, kun $\lambda_0 +\lambda_1 =1$ ja $a+\lambda_2=1$, eli kun $a\lambda_0+a\lambda_1+\lambda_2=a(\lambda_0+\lambda_1)+\lambda_2=a+\lambda_2=1$. 
Oletetaan, että jollain $n\geq 1$ 
\begin{equation*}
\lambda_0 x_0+\lambda_1 x_1+\cdots+\lambda_{n} x_{n}\in C(A), \text{kun } \lambda_0+\lambda_1+\cdots+ \lambda_{n} =1.
 \end{equation*}
%\begin{equation*}\begin{split} &x_1+\lambda_2 x_2+\cdots+\lambda_{n-1} x_{n-1}\in C(A), \\ &\text{kun } \lambda_0+\lambda_1+\cdots+ \lambda_{n-1} =1. \end{split} \end{equation*}
Tällöin myös
\begin{equation*}
\begin{split}
&a(\lambda_0 x_0+\lambda_1 x_1+\cdots+\lambda_{n}x_n)+\lambda_{n+1} x_{n+1}\\=&a\lambda_0 x_0+a\lambda_1 x_1+\cdots+a\lambda_{n}x_n+\lambda_{n+1} x_{n+1}
\in C(A),
\end{split}
\end{equation*}
%\text{ kun }a+ \lambda_{n+1} =1.
%\end{equation*}
%kun $a+ \lambda_{n+1} =1$, eli 
kun 
\begin{equation*}
\begin{split}
a\lambda_0+a\lambda_1+\cdots+ a\lambda_{n}+ \lambda_{n+1}=&a(\lambda_0+\lambda_1+\cdots+ \lambda_{n})+ \lambda_{n+1}\\=&a(1)+ \lambda_{n+1}=a+ \lambda_{n+1}=1.
\end{split}
\end{equation*} 
Siis konveksi verho $C(A)$ sisältää kaikki lineaarikombinaatiot joukon $A$ vektoreista.
%Tällöin koska $f$ kuuluu konveksiin verhoon $C(hX)$, niin $f$ on lineaarikombinaatio vektorialiavaruuden $hX$ vektoreista. Tällöin olkoon $a_0,a_1,\cdots, a_k \in A$, $\lambda_0,\lambda_1,\cdots, \lambda_k \in \R$ ja $\lambda_i\geq 0$, kaikilla $ i\in\N$ niin, että $$\sum_{i=0}^n \in A \lambda_i f_{a_i}\qquad \text{ missä } \sum_{i=0}^n \lambda_i =1.$$
%Tällöin $C(A)=\bigcap_{j\in J} A_j$ on pienin konveksi joukko, jolla $A\subset C(A)$, jolloin $C(A)$ on joukon $A$ konveksi verho.\\
%convex hull = kaikkien A:n sisältävien konveksien joukkojen leikkaus -> konveksi ja sisältää A:n.


%\begin{maar}

%\end{maar}

\newpage
\section{Banachin avaruus}

\begin{maar}Cauchyn jono \emph{
Jono $(x_n\colon n\in \N$
}
\end{maar}

\begin{maar}Banachin avaruus on täydellinen normiavaruus.
\end{maar}

\begin{maar}Separoituva avaruus. \emph{Olkoon $X$ metrinen avaruus ja $\mathcal{B}\subset \mathcal{P}( X)$ epätyhjien avointen osajoukkojen perhe. Tällöin avaruus $X$ on separoituva, jos seuraava ehto pätee: On olemassa numeroituva tiheä osajoukko $\{a_0,a_1,\cdots\}=A\subset X$ niin, että jokaista $B\in\mathcal{B}$ kohti löytyy ainakin yksi $a_i\in A$, jolla $a_i\in B$ jollain $i\in \N$.}
\end{maar}


\newpage
\section{Kuratowskin upotuslause}
\noindent 
\begin{teoreema}
Jokaista metristä avaruutta $(X,d)$ kohti on olemassa normiavaruus $Z$ ja upotus $h\colon X\rightarrow Z$ missä $ hX\subset Z$ on suljettu konveksissa verhossa $C(hX)$. 
\end{teoreema} 
 
\vspace{0.5cm}

\noindent\emph{Todistus.} Olkoon $(X,d)$ metrinen avaruus. Tällöin myös
$$d'(x,y)=\dfrac{d(x,y)}{1+d(x,y)} \, , \qquad x,y\in X$$
on metriikka joukossa $X$, sillä kaikilla $x,y,z\in X$:
\begin{itemize}
\item[(M1)]$d'(x,z)=\dfrac{d(x,z)}{1+d(x,z)}\leq \dfrac{d(x,y)+d(y,z)}{1+d(x,y)+d(y,z)}$
$$= \dfrac{d(x,y)}{1+d(x,y)+d(y,z)}+ \dfrac{d(y,z)}{1+d(x,y)+d(y,z)}$$
$$\leq \dfrac{d(x,y)}{1+d(x,y)}+ \dfrac{d(y,z)}{1+d(y,z)}=d'(x,y)+d'(y,z)$$
\item[(M2)]$d'(x,y)=\dfrac{d(x,y)}{1+d(x,y)}=\dfrac{d(y,x)}{1+d(y,x)}=d'(y,x)$
\item[(M3)]$d'(x,y)=\dfrac{d(x,y)}{1+d(x,y)}=0$, jos ja vain jos $d(x,y)=0$, eli jos ja vain jos $x=y$.
\end{itemize} 
Tällöin joukon $X$ pisteiden $d'$-etäisyydet toisistaan ovat korkeintaan $1$.

Olkoon $Z$ joukon $X$ kaikkien rajoitettujen jatkuvien funktioiden joukko. Asetetaan
$$|f_1|=\sup_{x\in X} |f_1(x)| \, , \qquad f_1\in Z$$
ja
$$d(f_1,f_2)=|f_1-f_2| = \sup_{x\in X} |f_1 (x)-f_2 (x)| \, , \qquad f_1, f_2 \in Z .$$

Tällöin $Z$ on normiavaruus, sillä kaikilla $g_1,g_2\in Z, a\in \R$:
\begin{itemize}
\item[(N1)] $|g_1+g_2|=\sup_{x\in X} |g_1(x)+g_2(x)|\\
\phantom{|g_1+g_2|} \leq \sup_{x\in X} |g_1(x)|+\sup_{x\in X} |g_2(x)|=|g_1|+|g_2|$
%\[\begin{split}|g_1+g_2|=&\sup_{x\in X} |g_1(x)+g_2(x)|\\\leq &\sup_{x\in X} |g_1(x)|+\sup_{x\in X} |g_2(x)|=|g_1|+|g_2|\end{split}\]
\item[(N2)] $|ag_1|=\sup_{x\in X} |ag_1(x)|=\sup_{x\in X} (|a||g_1(x)|)\\
\phantom{|a_{1ag}} =|a|\sup_{x\in X} |g_1(x)|=|a||g_1|$
\item[(N3)] $|g_1|=\sup_{x\in X} |g_1(x)|=0\implies g_1(x)=0\qquad$ kaikilla $ x\in X$
\end{itemize}
Seuraavaksi määritellään homeomorfismi $h\colon X\rightarrow h(X)\subset Z$. Tätä varten asetamme funktion $f_x\in Z$ jokaiselle $x\in X$ yhtälön $f_x (y)=d(x,y)$ mukaisesti, eli
%Asetamme 
$$h(x)=f_x\, ,\qquad x\in X. $$ 
Tällöin pätee 
$$d(f_{x_1} ,f_{x_2} ) \geq |d(x_1 , x_2 )-d(x_2 , x_2 )|=d(x_1 ,x_2 ).$$ 
Toisaalta mielivaltaiselle $y\in X$ pätee
$$|f_{x_1} (y), f_{x_2} (y)|=|d(x_1 ,y )-d(x_2 ,y)|\leq d(x_1 , x_2 ),$$
jolloin siis $d(f_{x_1},f_{x_2})\leq d(x_1 , x_2 )$. Edeltävistä epäyhtälöistä saadaan $d(f_{x_1},f_{x_2})= d(x_1 , x_2 )$, joka osoittaa, että kuvaus $h$ säilyttää pisteiden väliset etäisyydet ja on siten homeomorfismi.\\
%\newpage
Osoitetaan, että $hX$ on suljettu konveksissa verhossa $C(hX)$. Tavoitteena on osoittaa, että minkä tahansa jonon $f_{x_n}\in hX$ raja-arvo kuuluu joukkoon $ hX$. Olkoon $f\in C(hX)$ ja määritellään 
$$f=\lim_{n\rightarrow\infty} f_{x_n} \, , \qquad f_{x_n}\in hX.$$ 
Tällöin koska $f$ kuuluu konveksiin verhoon $C(hX)$, niin $f$ on lineaarikombinaatio vektorialiavaruuden $hX$ vektoreista. Tällöin olkoon $a_0,a_1,\cdots, a_k \in X$, $\lambda_0,\lambda_1,\cdots, \lambda_k \in \R$ ja $\lambda_i\geq 0$, kaikilla $ i\in\N$ niin, että $$f=\sum_{i=0}^k \lambda_i f_{a_i}\qquad \text{ missä } \sum_{i=0}^k \lambda_i =1.$$
Tällöin on ainakin yksi $i\leq k$, jolla $\lambda_i \geq 1/(k+1)$. Tämä seuraa siitä, että $$\sum_{i=0}^k \frac{1}{k+1}=k\cdot \dfrac{1}{k+1}=\dfrac{k}{k+1}<1 .$$ Tällöin voidaan vaihtaa $\lambda_0$ ja $\lambda_i$ keskenään ja vastaavasti $a_0$ ja $a_i$ keskenään, jolloin saadaan $\lambda_0\geq 1/(k+1)$. Jos jollain $i\neq j$ ja $i,j\leq k$ %ja $i,j\in \N$ 
pätee $ a_i=a_j$, niin on mahdollista valita uudet $$a_0,a_1,\cdots a_i,\cdots, a_{j-1},a_{j+1},\cdots a_{k}$$ ja vastaavat $$\lambda_0,\lambda_1,\cdots, (\lambda_i+\lambda_j),\cdots ,
\lambda_{j-1},
\lambda_{j+1},\cdots, \lambda_{k-1}.$$ 
Voidaan siis valita sellaiset $a_0,a_1,\cdots, a_k$, jotka ovat erillisiä ja sellaiset $\lambda_0,\lambda_1,\cdots \lambda_k$, että $\lambda_0$ täyttää ehdon $\lambda_0\geq 1/(k+1)$. Tällöin $$d(f,f_{x_n})\geq |f(x_n)-f_{x_n}(x_n)|=|f(x_n)|\geq \lambda_0 f_{a_0}(x_n)\geq \dfrac{1}{k+1}d(a_0,x_n).$$
Tällöin yhtälöstä $\lim_{n\rightarrow\infty} f_{x_n}=f$ seuraa, että $\lim_{n\rightarrow\infty }x_n=a_0$. Siis $ f=f_{a_0}\in hX$ pätee.


%\newpage
%\section{Fréchetin upotuslause.} %Jokainen separoituva metrinen avaruus $(X,d)$ on upotettavissa (isometrisesti) Banachin avaruuteen $l^\infty$. 
%\begin{teoreema}
%Olkoon $Z$ rajoitettujen jonojen sup-normilla varustettu vektoriavaruus. Nyt jokaista separoituvaa metristä avaruutta $(X,d)$ kohti on olemassa upotus $h\colon X\rightarrow Z$. 
%\end{teoreema}
%\vspace{0.5cm}
%
%\emph{Todistus.} Olkoon $Z$ kaikkien rajoitettujen jonojen joukko. Asetetaan
%$$|f_1|=\sup_{x\in Z} |f_1(x)| \, , \qquad f_1\in Z$$
%ja
%$$d(f_1,f_2)=|f_1-f_2| = \sup_{x\in Z} |f_1 (x)-f_2 (x)| \, , \qquad f_1, f_2 \in Z .$$
%
%Olkoon $(X,d)$ separoituva metrinen avaruus. Tällöin on olemassa separoituvuuden määritelmän mukainen tiheä ja numeroituva osajoukko $\{x_0,x_1\cdots \}\subset X$. Tällöin voidaan määritellään 
%$$x\mapsto s^x,\quad s^x_i=d(x,x_i)-d(x_i,x_0).$$
%%Tällöin $d(x,x_i)-d(x_i,x_0)\leq d(x,x_0)$ ja 
%Tällöin Kuratowskin upotuslauseen antama laajennus kiinnitetyllä pisteellä $x_0$ tiheästä osajoukosta täydelliseen avaruuteen on
%$$\{x_0,x_1,\cdots\}\hookrightarrow L^\infty (\{x_0,x_1,\cdots\})\simeq l^\infty. $$
%Upotus $x\mapsto s^x$ säilyttää etäisyydet, sillä 
%$$|s^a_i-s^b_i|=|d(a,x_i)-d(b,x_i)|\leq d(a,b)\quad \text{kaikilla } a,b\in X,i\in .$$
%Yhtälön yhtäsuuruus pätee jos $x_i=a$ tai $x_i=b$. 



\newpage
\section{Kuratowskin upotuslause v2}
\begin{teoreema}Jokaista metristä avaruutta $X$ kohti on olemassa upotus Banachin avaruuteen $L^{\infty}(X)$ missä $L^{\infty}(X)$ on avaruuden $X$ kaikkien rajoitettujen funktioiden avaruus varustettuna sup-normilla.
\end{teoreema}

\emph{Todistus} Olkoon $x_0\in X$ ja 
$$ x\mapsto s^x,\quad s^x(a)=d(x,a)-d(a,x_0),x\in X.$$ 
Tällöin kolmioepäyhtälöllä saadaan 
$$|s^x(a)|\leq d(x,x_0)$$
ja 
$$|s^x(a)-s^y(a)|=|d(x,a)-d(y,a)|\leq d(x,y),$$
jossa yhtäsuuruus pätee, jos $a=x $ tai $a=y $.




\newpage
%Alkuperäinen muoto:\\
\section{Fréchetin upotuslause.} Jokainen separoituva metrinen avaruus $(X,d)$ on upotettavissa (isometrisesti) Banachin avaruuteen $l^\infty$. 
\begin{teoreema}Jokaista separoituvaa metristä avaruutta $(X,d)$ kohti on olemassa etäisyydet säilyttävä upotus Banachin avaruuteen $s\colon X\rightarrow l^\infty$. 
(Huomautus: Banachin avaruus $l^\infty=L^\infty(\N)$ on rajoitettujen jonojen sup-normilla varustettu avaruus.)
\end{teoreema}

\vspace{0.5cm}

\emph{Todistus.} Olkoon $(X,d)$ separoituva metrinen avaruus. Tällöin on olemassa separoituvuuden määritelmän mukainen tiheä ja numeroituva osajoukko $\{x_0,x_1\cdots \}\subset X$. Tällöin voidaan määritellään 
$$x\mapsto s^x,\quad s^x_i=d(x,x_i)-d(x_i,x_0).$$
Tällöin Kuratowskin upotuslauseen antama laajennus kiinnitetyllä pisteellä $x_0$ tiheästä osajoukosta täydelliseen avaruuteen on
$$\{x_0,x_1,\cdots\}\hookrightarrow L^\infty (\{x_0,x_1,\cdots\})\simeq l^\infty. $$
Upotus $x\mapsto s^x$ säilyttää etäisyydet, sillä 
$$|s^a_i-s^b_i|=|d(a,x_i)-d(b,x_i)|\leq d(a,b)\quad \text{kaikilla } a,b\in X,i\in .$$
Yhtälön yhtäsuuruus pätee jos $x_i=a$ tai $x_i=b$. 



%\vspace{1.5cm}
%[idea kirjoittaa kuratowskin upotuslauseen todistus auki. erityisesti keskittyen convex hull -osioon. Tarvittavat esitiedot alkuun, eli selitettynä tarvittavat käsitteet ja niiden todistukset]


 
\end{document}
