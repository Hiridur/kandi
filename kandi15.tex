\documentclass[12pt,a4paper,leqno]{report}

%\usepackage[ansinew]{inputenc}
\usepackage[utf8]{inputenc}
\usepackage[T1]{fontenc}
\usepackage[finnish]{babel}
\usepackage{amsthm}
\usepackage{amsfonts}         
\usepackage{amsmath}
\usepackage{amssymb}

\newcommand{\R}{\mathbb{R}}
\newcommand{\C}{\mathbb{C}}
\newcommand{\Q}{\mathbb{Q}}
\newcommand{\N}{\mathbb{N}}
\newcommand{\No}{\mathbb{N}_0}
\newcommand{\Z}{\mathbb{Z}}
\newcommand{\diam}{\operatorname{diam}}

\theoremstyle{plain}
\newtheorem{lause}[equation]{Lause}
\newtheorem{lem}[equation]{Lemma}
\newtheorem{prop}[equation]{Propositio}
\newtheorem{kor}[equation]{Korollaari}

\theoremstyle{definition}
\newtheorem{maar}[equation]{Määritelmä}
\newtheorem{konj}[equation]{Konjektuuri}
\newtheorem{esim}[equation]{Esimerkki}

\theoremstyle{remark}
\newtheorem{huom}[equation]{Huomautus}

\pagestyle{plain}
\setcounter{page}{1}
\addtolength{\hoffset}{-1.15cm}
\addtolength{\textwidth}{2.3cm}
\addtolength{\voffset}{0.45cm}
\addtolength{\textheight}{-0.9cm}

\title{Kuratowskin upotuslause}
\author{Pekka Keipi}
%\date{}

\begin{document}

\maketitle

\tableofcontents

\chapter{Johdanto}\label{johd}

Tämän tutkielman tarkoituksena on todistaa Kuratowskin upotuslause. Lause on nimetty puolalaisen matemaatikon Kazimierz Kuratowskin mukaan. Osa lauseen todistuksesta on tehtävänä kurssikirjassa Topologia 1, joka on yksi matematiikan pääaineopiskelijoiden aineopintoja.
\\
\\
Kazimierz Kuratowski (2.2.1896 - 18.6.1980) oli puolalainen matemaatikko, joka keskittyi tutkimuksessaan abstraktiin topologiaan ja metrisiin struktuureihin. 
 jne.. Luku \ref{normi}, ks.\ \cite{Topo1}
%Tämän tutkielman tarkoituksena on esitellä todennäköisyys- ja momenttigeneroivia
%funktioita. Yleisesti generoiva funktio on hyödyllinen esimerkiksi kombinatoriikassa. 
%Matematiikan yliopisto-opinnoissa generoivat funktiot tulevat yleensä ensimmäisen
%kerran tutuiksi todennäköisyyslaskennan kursseilla.

\chapter{Funktion rajoittuma, normi ja sup-normi}\label{normi}

Tässä kappaleessa esitellään vektori- ja normiavaruuksien ja näiden funktioiden perusominaisuuksia. \cite{Topo1}
\begin{maar}
\emph{Vektoriavaruus}\\
Joukko $V$ on $\R$-kertoiminen vektoriavaruus, jos kaikkiin $v,w\in V$ ja $a\in \R$ on liitetty yksikäsitteinen summa $v+w\in V$ ja tulo $av\in V$ niin, että seuraavat ominaisuudet ovat voimassa:
\begin{itemize}
\item[i)\phantom{iiv}] $(u+v)+w=u+(v+w)\text{ kaikilla }u,v,w\in V.$
\item[ii)\phantom{iv}] $v+w=w+v\text{ kaikilla }v,w\in V.$
\item[iii)\phantom{v}] On olemassa sellainen $0=0_{v},$ että $v+0=v\text{ kaikilla }v\in V.$
\item[iv)\phantom{ii}] Jokaiseen $v\in V $ liittyy sellainen $-v\in V$, että $v+(-v)=0$.
\item[v)\phantom{iii}] $a(v+w)=av+aw\text{ kaikilla }a\in\R, v,w\in V.$
\item[vi)\phantom{ii}] $(a+b)v=av+bv\text{ kaikilla }a,b\in\R, v\in V.$
\item[vii)\phantom{i}] $a(bv)=(ab)v\text{ kaikilla }a,b\in\R, v\in V.$
\item[viii)] $1_{v}=v$ kaikilla $v\in V.$
\end{itemize}
Tässä avaruuden $V$ alkioita kutsutaan vektoreiksi ja avaruuden $\R$ alkioita skalaareiksi.

\end{maar}
%\begin{esim}Tärkein esimerkki vektoriavaruuksista on euklidinen avaruus $\R^n$, jossa luku $n\in \N$ on avaruuden ulottuvuus.\end{esim}
\begin{maar}
\emph{Vektorialiavaruus}\\
Osajoukko $W\subset V$ on vektoriavaruuden $V$ (vektori)aliavaruus, jos
\begin{itemize}
\item[i)\phantom{iiv}] $v+w\in W\text{ kaikilla }v,w\in W,$
\item[ii)\phantom{iv}] $av\in W\text{ kaikilla }a\in\R, v\in W$ ja
\item[iii)\phantom{v}] $0_{v}\in W$
\end{itemize}

\end{maar}
\begin{maar}
\emph{Rajoitettu funktio}\\
Olkoon $D$ avaruus ja $F(D,\R)$ kaikkien avaruudessa $D$ määriteltyjen 
reaaliarvoisten funktioiden $f\colon D\rightarrow \R$ joukko. Joukko $F(D,\R)$ on vektoriavaruus \cite{Topo1}. %luku 1.1.
Funktio $f\colon D\rightarrow \R$ on rajoitettu, 
jos on olemassa sellainen $M\in\R,M\geq 0$, 
jolla $|f(x)|\leq M$ kaikilla $x\in D$.\\\\
%Rajoitettujen funktioiden avaruutta $\cup f, f\colon  $
Rajoitettujen funktioiden avaruus $Raj(D,\R )$ koostuu kaikista rajoitetuista funktioista $f\colon D\rightarrow \R $.

\end{maar}
\begin{lem}
%Olkoon $D$ avaruus ja $F(D,\R)$ avaruuden $D$ reaaliarvoisten funktioiden avaruus. Tällöin r
Rajoitettujen funktioiden avaruus $Raj(D,\R )$ on reaaliarvoisten funktioiden vektoriavaruuden $F(D,\R)$ aliavaruus.
\end{lem}
%\noindent
\begin{proof}
 Olkoon $f,g\in Raj(D,\R),M,N\in\R, M,N\geq 0$ niin, että $|f(x)|\leq M$ ja $|g(x)|\leq N$ kaikilla $x\in D$. Tällöin seuraavat kohdat pätevät: 
\begin{itemize}
\item[i)\phantom{iiv}] $|f(x)+g(x)|\leq|f(x)|+|g(x)|\leq M+N\leq\infty$, siis $f+g\in Raj(D,\R)$ kaikilla $f,g\in Raj(D,\R),$
\item[ii)\phantom{iv}] $|af(x)|= |a|\cdot|f(x)|\leq |a|\cdot M\leq\infty$, siis $af\in Raj(D,\R)$ kaikilla $a\in\R, f\in Raj(D,\R)$ ja
\item[iii)\phantom{v}] $0_{F(D,\R)}(x)=0$ kaikilla $x\in D$, jolloin $0_{F(D,\R)} \in Raj(D,\R )$.
\end{itemize}
\end{proof}

\begin{maar} \emph{Normi}\\
Olkoon $E$ vektoriavaruus ja $|\cdot|\colon E\rightarrow \R_+$, $x\mapsto|x|$ kuvaus joukossa $E$. %Kuvaus $|\cdot|$ liittää jokaiseen joukon $E$ alkioon $x\in E$ reaaliluvun $|x|\geq 0$.
Kuvaus $|\cdot|$ on normi avaruudessa $E$, jos seuraavat ominaisuudet pätevät kaikilla $ x,y\in E, a\in\R$.
\begin{itemize}
\item[(N1)]$|x+y|\leq |x|+|y|$,
\item[(N2)]$|ax|=|a||x|$,
\item[(N3)]Jos $|x|=0$, niin $x=\bar{0}$.
\end{itemize} 
Vektoriavaruutta, jossa on annettu jokin normi, sanotaan normiavaruudeksi.

\end{maar} 

\begin{esim}
Määritellään joukossa $\R ^n$ tavallinen euklidinen normi $|x| = \sqrt{x^{2}_{1} + \cdots +x^{2}_{n}}$.
%Olkoon $\R ^n$ joukko, jossa määritellään tavallinen euklidinen normi $|x| = \sqrt{x^{2}_{1} + \cdots +x^{2}_{n}}$.

\end{esim}

\begin{maar}\label{sup-normi} \emph{Sup-normi}\\
Olkoon $D$ epätyhjä joukko ja $Raj(D,\R )$ kaikkien avaruudessa $D$ määriteltyjen rajoitettujen funktioiden $f\colon D\rightarrow \R$ vektoriavaruus. Yhtälö $||f||=\sup\{|f(x)|\colon x\in D\}$ määrittelee normin avaruudessa $Raj(D,\R )$. %Normia $||\cdot||$ 
Tätä normia sanotaan avaruuden $Raj(D,\R )$ sup-normiksi.

\end{maar} 

\begin{proof}
\begin{itemize}
\item[(N1)] Olkoon $f,g\in Raj(D,\R )$ ja $x\in D$. Tällöin
\begin{equation*}
|f+g|(x)=|f(x)+g(x)|\leq|f(x)|+|g(x)|\leq||f||+||g||
\end{equation*}
kaikilla $x\in D$, joten $||f+g||=\sup\{|f(x)+g(x)|\}\leq||f||+||g||$.

\item[(N2)] Olkoon $f\in Raj(D,\R )$ ja $x\in D$. Tällöin
\begin{equation*}
%\label{N2todistus1}
|af|(x)=|af(x)|=|a||f(x)|\leq|a|||f||
\end{equation*}
kaikilla $x\in D$, joten $||af||\leq|a|||f||$. Jos $a=0$, niin (N2) pätee muodossa $0=0$. %$$||af||=||0\cdot f||=0=|0|||f||=|a|||f||.$$ 
Jos $a\neq 0$, niin $f=a^{-1}af$ ja edellisen
%yhtälön $\ref{N2todistus1}$ 
mukaan $||f||\leq|a^{-1}|||af||$
%$$|f|(x)=|a^{-1}af|(x)=|a^{-1}af(x)|=|a^{-1}||af(x)|\leq|a^{-1}|||af(x)||.$$
ja edelleen $||af||\geq|a|||f||$ kaikilla $x\in D$. Tällöin siis $||af||=|a|||f||$.

\item[(N3)]Jos $||f||=0$, niin $|f(x)|=0$ kaikilla $x\in D$, eli  $f=\bar{0}$.
\end{itemize} 
\end{proof}

\begin{esim}
Tapauksessa $D=\N$ saadaan kaikkien rajoitettujen jonojen joukko $raj(\N,\R).$ Joukon alkioita ovat rajoitetut jonot $x=(x_1,x_2,\dots).$ Tälle joukolle käytetään usein merkintää $l_\infty$.

\end{esim}


\chapter{Metrinen avaruus}\label{metrinen}
Tässä luvussa esitellään metristen avaruuksien ominaisuuksia. Metrisiä avaruuksia käsitellään syvemmin kirjassa Topologia I% \cite{Topo1} 
\begin{maar} \emph{Metrinen avaruus} 
on pari $(X,d)$, jossa $X$ on joukko ja $d$ on \emph{metriikka} joukossa $X$.
Tällöin $d\colon X\times X\rightarrow \R_+$ on kuvaus, jolle pätee seuraavat ominaisuudet kaikilla $x,y,z\in X$:
\begin{itemize}
\item[(M1)]$d(x,z)\leq d(x,y)+d(y,z)$
\item[(M2)]$d(x,y)=d(y,x)$
\item[(M3)]$d(x,y)=0$, jos ja vain jos $x=y$.
\end{itemize} 
\end{maar} 

\begin{esim}
Euklidinen metriikka avaruuden $\R^n$ kahden pisteen $p=(p_1,p_2,\dots,p_n)$ ja $q=(q_1,q_2,\dots,q_n)$ välillä on määritelty 
$$d(p,q)=\sqrt{(p_1-q_1)^2+(p_2-q_2)^2+\dots+(p_n-q_n)^2}.$$
\end{esim}

\begin{esim}
Normiavaruus on aina metrinen avaruus. 

\begin{proof}
Olkoon $E$ normiavaruus ja $x,y\in E$. 
Metriikaksi voidaan valita $d(x,y)=|x-y|$, jolloin kaikilla $x,y\in E$ pätee
\begin{itemize}
\item[(M1)]%\begin{align*}
$d(x,z)=|x-z|=|x-y+y-z|=|(x-y)+(y-z)|\leq|x-y|+|y-z|=d(x,y)+d(y,z)$
%\end{align*}
\item[(M2)]$d(x,y)=|x-y|=|y-x|=d(y,x)$
\item[(M3)]$d(x,y)=|x-y|=0$, jos ja vain jos $x=y$.
\end{itemize}
\end{proof}

\end{esim}
\begin{maar}\emph{Joukkojen välinen etäisyys}\\
Olkoon $(X,d)$ metrinen avaruus ja $A,B\subset X$. Tällöin etäisyys osajoukkojen $A$ ja $B$ välillä on määritelty $d(A,B)=\inf\{d(a,b)\colon a\in A, b\in B\}$, eli etäisyyksien $d(a,b)$ suurin alaraja, kun $a\in A$ ja $b\in B$. \\
\\
Tällöin pätee $d(A,B)\geq 0$ ja jos $A= \emptyset$ tai $B= \emptyset$, niin $d(A,B)= 0$. 

\end{maar}
\begin{lause}
Olkoon $(X,d)$ metrinen avaruus, $x,y\in X$ ja $A\subset X$ epätyhjä. Tällöin $|d(x,A)-d(y,A)|\leq d(x,y)$. Erityisesti $|d(x,z)-d(y,z)|\leq d(x,y)$ kaikilla $z\in X$.
\end{lause}

\begin{proof}
 %Aputulos: $b+\inf\{a\in A\}=\inf\{b+a\colon a\in A\}$\\
Olkoon $a\in A$. Tällöin kaikilla $y\in A$ pätee $d(x,A)\leq d(x,a)\leq d(x,y)+d(y,a)$. Ottamalla infimum kaikkien $a\in A$ yli saadaan $d(x,A)\leq d(x,y)+d(y,A)$
%\begin{equation*}\begin{split} d(x,A)=&\inf\{d(x,b)\colon b\in A\}\leq\inf\{d(x,y)+d(y,b)\colon b\in A\}\\=&d(x,y)+\inf\{d(y,b)\colon b\in A\}=d(x,y)+d(y,A)\end{split}\end{equation*}
ja edelleen $d(x,A)-d(y,A)\leq d(x,y) $.
Vastaavasti olkoon $y\in A$, jolloin kaikilla $x\in X$ pätee $d(y,A)-d(x,A)\leq d(y,x)=d(x,y) $. Nyt $d(x,A)-d(y,A)\leq d(x,y) $ ja $d(y,A)-d(x,A)\leq d(x,y) $, jolloin siis $|d(x,A)-d(y,A)|\leq d(x,y) $.

\end{proof}

\begin{maar} \emph{Homeomorfismi} 
tarkoittaa kuvausta $f\colon X\rightarrow Y$, jolla 
\begin{itemize}
\item[(H1)] $f$ on bijektio,
\item[(H2)] $f$ on jatkuva,
\item[(H3)] $f^{-1} \colon Y\rightarrow X$ on jatkuva. 
\end{itemize}
\end{maar}

\begin{maar}\emph{Upotus} 
tarkoittaa kuvausta $f\colon X\rightarrow Y$, joka määrittelee homeomorfismin $f_1\colon X\rightarrow fX$, jolla $f_1(x)=f(x)$ kaikilla $x\in X$. %Tällöin funktiolla $f$ pätee seuraavat ehdot:
%\begin{itemize}
%\item[(1)] $f$ on injektio,
%\item[(2)] $f$ on jatkuva,
%\item[(3)] $f^{-1} \colon f[X]\rightarrow X$ on jatkuva. 
%\end{itemize}
\end{maar}
\begin{maar}\emph{Isometria}
 on etäisyydet säilyttävä kuvaus. Olkoon $(X,d)$ ja $(X',d')$ metrisiä avaruuksia. %Avaruudet $(X,d)$ ja $(X',d')$ ovat isometrisiä,
Tällöin kuvaus $h\colon X\rightarrow X'$ on isometria, jos $d(x_1,x_2)=d'(h(x_1),h(x_2))$ kaikilla $x_1,x_2 \in X$.
\end{maar}
%\begin{lem}
%Isometria on aina myös upotus. 
%
%\end{lem}
%\begin{proof} Olkoon $(X,d)$ ja $(X',d')$ metrisiä avaruuksia ja kuvaus $h\colon X\rightarrow X'$ isometria. Tällöin $d(x_1,x_2)=d'(h(x_1),h(x_2))$ kaikilla $x_1,x_2 \in X$.
%\begin{itemize}
%\item[(1)] $h$ on bijektio,
%\item[(2)] $h$ on jatkuva,
%\item[(3)] $h^{-1} \colon X'\rightarrow X$ on jatkuva. 
%\end{itemize}
%\end{proof}
\begin{esim}
Peilaus, rotaatio ja siirtokuvaus ovat geometriasta tuttuja isometrioita avaruudessa $\R^2$.
\end{esim}
\begin{lem}\label{isolemma} Bijektiivinen isometria on aina homeomorfismi.
\end{lem}
\begin{proof}
Olkoon $(X,d)$ ja $(X',d')$ metrisiä avaruuksia ja olkoon kuvaus $h\colon X\rightarrow X'$ isometria. Tällöin $d(x_1,x_2)=d'(h(x_1),h(x_2))$ kaikilla $x_1,x_2 \in X$. Tällöin
\begin{itemize}
\item[(H1)] $h$ on määritelmän nojalla bijektio,
\item[(H2)] Kaikilla $\epsilon>0$ on olemassa $\delta>0$, jolla pätee: Jos $d(x_1,x_2)<\epsilon$, niin $d'(h(x_1),h(x_2))=d(x_1,x_2)<\epsilon=\delta$. Siis $h$ on jatkuva.
\item[(H3)] Edellisen kohdan ja isometrisuuden nojalla myös $f^{-1} \colon Y\rightarrow X$ on jatkuva. 
\end{itemize}
\end{proof}
\begin{lem}\label{rajmetri} Olkoon $(X,d)$ metrinen avaruus. Tällöin
$$d'\colon X\times X \rightarrow \R,\quad d'(x,y)=\dfrac{d(x,y)}{1+d(x,y)} \, , \qquad x,y\in X$$
on metriikka joukossa $X$ ja joukon $X$ pisteiden $d'$-etäisyydet toisistaan ovat korkeintaan $1$.
\end{lem}
\begin{proof}
Olkoon $(X,d)$ metrinen avaruus. Kaikilla $x,y,z\in X$ pätee
\begin{itemize}
\item[(M1)]$d'(x,z)=\dfrac{d(x,z)}{1+d(x,z)}
\leq \dfrac{d(x,y)+d(y,z)}{1+d(x,y)+d(y,z)}$\\
$\phantom{d'(x,z)=\dfrac{d(x,z)}{1+d(x,z)}}
= \dfrac{d(x,y)}{1+d(x,y)+d(y,z)}+ \dfrac{d(y,z)}{1+d(x,y)+d(y,z)}$\\
$\phantom{d'(x,z)=\dfrac{d(x,z)}{1+d(x,z)}}
\leq \dfrac{d(x,y)}{1+d(x,y)}+ \dfrac{d(y,z)}{1+d(y,z)}=d'(x,y)+d'(y,z),$
\item[(M2)]$d'(x,y)=\dfrac{d(x,y)}{1+d(x,y)}
=\dfrac{d(y,x)}{1+d(y,x)}=d'(y,x)$ ja
\item[(M3)]$d'(x,y)=\dfrac{d(x,y)}{1+d(x,y)}=0$, jos ja vain jos $d(x,y)=0$, eli jos ja vain jos $x=y$.
\end{itemize} 
Joukon $X$ pisteiden $d'$-etäisyydet toisistaan ovat korkeintaan $1$, sillä $d(x,y)\leq 1+d(x,y)$ kaikilla $x,y\in X$.
\end{proof}

%\begin{lem}
%Olkoon $(X,d)$ metrinen avaruus. 
%Jokaista metristä avaruutta $(X,d)$ kohden 
%on olemassa sellainen rajoitettu kuvaus 
%$d'\colon X \rightarrow \R$, joka on homotooppinen metriikan $d$ kanssa. 
%\end{lem}
%\begin{lem}
%Metrisen avaruuden $(X,d)$ joukolle $X$ on aina olemassa sellainen metriikka $d'$, jolla joukon $X$ pisteiden $d'$-etäisyydet toisistaan ovat korkeintaan $1$.
%\end{lem}
%\begin{proof} 
%Olkoon $(X,d)$ metrinen avaruus. Tällöin myös
%$$d'(x,y)=\dfrac{d(x,y)}{1+d(x,y)} \, , \qquad x,y\in X$$
%on metriikka joukossa $X$, sillä kaikilla $x,y,z\in X$:
%Tällöin joukon $X$ pisteiden $d'$-etäisyydet toisistaan ovat korkeintaan $1$.
%\end{proof}

\chapter{Konveksi verho}\label{konverho}
Tässä luvussa esitellään konveksius, konveksi verho ja tämän ominaisuuksia. Näitä käsitteitä tarkastellaan syvemmin kirjoissa Topologia I ja Topologia II.
\begin{maar}%%Konveksi joukko sisältää kaikkien alkioidensa väliset janapolut. 
\emph{Konveksius.}
 Olkoon $E$ normiavaruus, $A\subset E$ ja $a,b\in A$. Yhtälö 
$$\alpha (t)=a+t(b-a)=(1-t)a+tb$$ 
määrittelee janapolun $\alpha \colon [0,1]\rightarrow E$. Kuvajoukko $\alpha [0,1] = [a,b]$ on jana, jonka päätepisteet ovat $\alpha (0)=a\in A$ ja $\alpha (1)=b\in A$. 
% Merkitään $\alpha I = [a,b]$. 
Joukko $A\subset E$ on konveksi, jos ja vain jos $[a,b]\subset A$ kaikilla $a,b\in A$.
%Jos $[a,b]\in A$, niin joukko $A$ on konveksi.%aina kun $a,b\in A$.
\end{maar}
\begin{esim}
%Avaruus $\R ^n$ on konveksi, sillä minkä tahansa kahden pisteen $a,b\in \R ^n $ välinen jana $[a,b]$ kuuluu avaruuteen $\R ^n$.
Avaruuden $\R ^n$ suljettu yksikkökuula $ \bar B(0,1)= \{x\in \R ^n \colon |x|\leq 1\}$ on konveksi.
\end{esim}
\begin{maar}\emph{Konveksi verho.} Olkoon $E$ normiavaruus, $A\subset E$ ja $(A_j)_{j\in J}\subset \mathcal{P}(E)$ 
kaikkien sellaisten konveksien osajoukkojen $A_k\subset E$, $k\in J$ muodostama perhe, joilla %joukko $A_k$ konveksi ja 
$A\subset A_k$% kaikilla $k\in J$
. Tällöin joukko $C(A)=\bigcap_{j\in J} A_j$ on joukon $A$ konveksi verho. 
\end{maar}
\begin{lem} Konveksi verho $C(A)$ on konveksi.\end{lem} 
\begin{proof} Olkoon $a,b\in C(A)=\bigcap_{j\in J} A_j$. 
Tällöin $a,b\in A_i$ kaikilla $i\in J$. %ja koska erityisesti 
Jokainen $A_i$ on konveksi, joten kaikilla $i\in J$ pätee $[a,b]\subset A_i$.
Tällöin kaikilla $a,b\in C(A)$ pätee $[a,b] \subset C(A)$, joten konveksi verho $C(A)$ on konveksi.
\end{proof}
%%Tästä seuraa, että konveksi verho $C(A)$ on konveksi. \end{kor}
\begin{kor}Konveksi verho $C(A)$ on pienin konveksi joukko, joka sisältää joukon $A$.
\end{kor}
%%Tällöin koska $C(A)=\bigcap_{j\in J} A_j$, niin $C(A)\subset A_i$ kaikilla $i\in J$. Tällöin konveksi verho $C(A)$ on pienin konveksi joukko, joka sisältää joukon $A$.
\begin{proof} Olkoon $B$ sellainen konveksi joukko, jolla $A\subset B\subset C(A)$. 
Joukko $B$ on konveksi ja $A\subset B$, joten $B\in (A_j)_{j\in J}$. 
Tällöin $$C(A)=\bigcap_{j\in J} A_j\subset B.$$
%, sillä $C(A)\subset A_i$ kaikilla $i\in J.$
Siis $B= C(A)$ ja $C(A)$ on pienin konveksi joukko, jolla $A\subset C(A)$.
\end{proof}

%Seuraavasta lemmasta on kaksi versiota, kumpi parempi?
\begin{lem}\label{konveksilineaari} Olkoon $E$ normiavaruus ja $A\subset E$. 
%Konveksi verho $C(A)$ on joukon $A$ vektorien muotoa $$\lambda_0 x_0+\lambda_1 x_1+\cdots+\lambda_{n} x_{n},\quad\text{missä } x_i\in A, \lambda_i\geq0\text{, ja }\lambda_0+\lambda_1+\cdots+ \lambda_{n} =1$$ olevien lineaarikombinaatioiden muodostama joukko. 
Konveksi verho $C(A)$ on muotoa 
$$\sum_{i=0}^n \lambda_i x_i,\qquad \text{ missä } x_i\in A, \lambda_i\geq0\text{ ja } \sum_{i=0}^n \lambda_i =1 $$ 
%$$\lambda_0 x_0+\lambda_1 x_1+\cdots+\lambda_{n} x_{n},\quad\text{missä } x_i\in A, \lambda_i\geq0\text{ ja }\lambda_0+\lambda_1+\cdots+ \lambda_{n} =1$$ 
%$\lambda_0 x_0+\lambda_1 x_1+\cdots+\lambda_{n} x_{n}$ 
olevien lineaarikombinaatioiden muodostama joukko.
%, missä $x_i\in A, \lambda_i\geq0\text{, ja }\lambda_0+\lambda_1+\cdots+ \lambda_{n} =1$. 
\end{lem}
%\begin{lem} Olkoon $E$ normiavaruus ja $A\subset E$. 
%Konveksi verho $C(A)$ on joukon $A$ vektorien muotoa 
%$$\lambda_0 x_0+\lambda_1 x_1+\cdots+\lambda_{n} x_{n},\quad\text{missä } x_i\in A, \lambda_i\geq0\text{, ja }\lambda_0+\lambda_1+\cdots+ \lambda_{n} =1$$ 
%olevien lineaarikombinaatioiden muodostama joukko. 
%%Konveksi verho $C(A)$ on joukon $A$ vektorien muotoa $\lambda_0 x_0+\lambda_1 x_1+\cdots+\lambda_{n} x_{n},$ olevien lineaarikombinaatioiden muodostama joukko, missä $x_i\in A, \lambda_i\geq0\text{, ja }\lambda_0+\lambda_1+\cdots+ \lambda_{n} =1$. 
%\end{lem}

\begin{proof}
Konveksi verho $C(A)$ on konveksi ja $A\subset C(A)$, 
joten kaikilla $a,b\in A$ pätee $[a,b]\subset C(A)$, 
toisin sanoen, $(1-t)a+tb\in C(A)$ kaikilla $0\leq t \leq 1$, 
erityisesti 
\begin{equation}\label{konvsis}
c_0 a+c_1 b \in C(A),\text{ kun }c_0,c_1\geq 0\text{ ja }c_0 +c_1=1.
\end{equation}
%Konveksi verho $C(A)$ on konveksi, joten kaikilla $x_0,x_1\in C(A)$ pätee $[x_0,x_1]\subset C(A)$, 
%toisin sanoen, $(1-t)x_0+tx_1\in A$ kaikilla $0\leq t \leq 1$. 
Olkoon $x_0,x_1,\dots, x_n, x_{n+1} \in A$, 
$d_0,d_1,\dots, d_n,e, \lambda_0 ,\lambda_1,\dots,\lambda_{n},\lambda_{n+1} \in \R$ 
ja $d_i,e, \lambda_{i}\geq 0$ kaikilla $ i\in\N$. 
%Tällöin erityisesti 
%$\lambda_0 x_0+\lambda_1 x_1 \in C(A)$, kun $\lambda_0 +\lambda_1=1$.
%\begin{equation}\lambda_0 x_0+\lambda_1 x_1 \in C(A),
%\text{ kun } x_0,x_1\in C(A), \lambda_0, \lambda_1\geq 0\text{ ja } \lambda_0 +\lambda_1=1.
%\end{equation}
Oletetaan, että jollain $n\geq 1$ pätee 
\begin{equation*}
d_0 x_0+d_1 x_1+\cdots+d_{n} x_{n}\in C(A), \text{kun } d_0+d_1+\cdots+ d_{n} =1.
 \end{equation*}
%
%
%
%
Olkoon $\lambda_0+\lambda_1+\cdots+ \lambda_{n}+ \lambda_{n+1} =1$. Tällöin
\begin{equation*}
\begin{split}
&\lambda_0 x_0+\lambda_1 x_1+\cdots+\lambda_{n} x_{n}+\lambda_{n+1} x_{n+1}\\
=&e\left(\frac{ \lambda_0}{e} x_0+\frac{ \lambda_1}{e} x_1+\cdots+\frac{ \lambda_n}{e} x_{n}\right) +\lambda_{n+1} x_{n+1}\in C(A)
\end{split}
%\quad\text{missä } x_i\in A, \lambda_i\geq0\text{, ja }
%\lambda_0+\lambda_1+\cdots+ \lambda_{n} =1
%\sum_{i=0}^{n}{\lambda_i}=1
\end{equation*}
%
%
%
Olkoon $\lambda_0+\lambda_1+\cdots+ \lambda_{n}+ \lambda_{n+1} =1$. Tällöin
\begin{equation*}
\begin{split}
&\lambda_0 x_0+\lambda_1 x_1+\cdots+\lambda_{n} x_{n}+\lambda_{n+1} x_{n+1}\\
=&(1-\lambda_{n+1})\left(\frac{ \lambda_0}{1-\lambda_{n+1}} x_0+\frac{ \lambda_1}{1-\lambda_{n+1}} x_1+\cdots+\frac{ \lambda_n}{1-\lambda_{n+1}} x_{n}\right) +\lambda_{n+1} x_{n+1}\in C(A)
\end{split}
%\quad\text{missä } x_i\in A, \lambda_i\geq0\text{, ja }
%\lambda_0+\lambda_1+\cdots+ \lambda_{n} =1
%\sum_{i=0}^{n}{\lambda_i}=1
\end{equation*}
%
%
%
%
Tällöin konveksiuden ja kaavan \ref{konvsis} nojalla pätee 
\begin{equation*}
\begin{split}
&e(d_0 x_0+d_1 x_1+\cdots+d_{n}x_n)+\lambda_{n+1} x_{n+1}\\
=&e d_0 x_0+e d_1 x_1+\cdots+e d_{n}x_n+\lambda_{n+1} x_{n+1}
\in C(A),
\end{split}
\end{equation*}
kun $e+\lambda_{n+1} =1$. Valitsemalla $ed_i= \lambda_{i}$ kaikilla $i\in {0,1,\dots,n}$ saadaan $$\lambda_0 x_0+\lambda_1 x_1+\cdots+\lambda_{n} x_{n}+ \lambda_{n+1} x_{n+1}
\in C(A).$$ Siis $C(A)$ sisältää kaikki muotoa 
\begin{equation*}
\lambda_0 x_0+\lambda_1 x_1+\cdots+\lambda_{n} x_{n},
\quad\text{missä } x_i\in A, \lambda_i\geq0\text{, ja }
%\lambda_0+\lambda_1+\cdots+ \lambda_{n} =1
\sum_{i=0}^{n}{\lambda_i}=1
\end{equation*} 
olevat lineaarikombinaatiot. 
%
%
%
%
Osoitetaan seuraavaksi, että $C(A)$ sisältää ainoastaan kyseisiä lineaarikombinaatiota. 

\end{proof}

%\begin{proof} Osoitetaan ensin, että kaikki kyseistä muotoa olevat lineaarikombinaatiot kuuluvat konveksiin peitteeseen.
%Olkoon $x_0,x_1,\cdots, x_n, x_{n+1} \in A$, $a,\lambda_0,\lambda_1,\cdots, \lambda_n, \lambda_{n+1} \in \R$ ja $a,\lambda_i\geq 0$ kaikilla $ i\in\N$. Konveksiuden nojalla 
%%$a x_1+b x_2 \in C(A)$, kun $a +b =1$. 
%$\lambda_0 x_0+\lambda_1 x_1 \in C(A)$, kun $\lambda_0 +\lambda_1=1$.
%%Edelleen $\lambda_1 (ax_1+b x_2)+\lambda_2 x_3 \in C(A)$, kun $\lambda_1 +\lambda_2 =1$ ja $a+b=1$. 
%%ja edelleen $a (\lambda_0 x_0+\lambda_1 x_1)+\lambda_2 x_2=a\lambda_0 x_0+a\lambda_1  x_1+\lambda_2 x_2 \in C(A)$, kun $\lambda_0 +\lambda_1 =1$ ja $a+\lambda_2=1$, eli kun $a\lambda_0+a\lambda_1+\lambda_2=a(\lambda_0+\lambda_1)+\lambda_2=a+\lambda_2=1$. 
%Oletetaan, että jollain $n\geq 1$ 
%\begin{equation*}
%\lambda_0 x_0+\lambda_1 x_1+\cdots+\lambda_{n} x_{n}\in C(A), \text{kun } \lambda_0+\lambda_1+\cdots+ \lambda_{n} =1.
% \end{equation*}
%%\begin{equation*}\begin{split} &x_1+\lambda_2 x_2+\cdots+\lambda_{n-1} x_{n-1}\in C(A), \\ &\text{kun } \lambda_0+\lambda_1+\cdots+ \lambda_{n-1} =1. \end{split} \end{equation*}
%Tällöin myös
%\begin{equation*}
%\begin{split}
%&a(\lambda_0 x_0+\lambda_1 x_1+\cdots+\lambda_{n}x_n)+\lambda_{n+1} x_{n+1}\\=&a\lambda_0 x_0+a\lambda_1 x_1+\cdots+a\lambda_{n}x_n+\lambda_{n+1} x_{n+1}
%\in C(A),
%\end{split}
%\end{equation*}
%%\text{ kun }a+ \lambda_{n+1} =1.
%%\end{equation*}
%%kun $a+ \lambda_{n+1} =1$, eli 
%kun 
%\begin{equation*}
%\begin{split}
%a\lambda_0+a\lambda_1+\cdots+ a\lambda_{n}+ \lambda_{n+1}=&a(\lambda_0+\lambda_1+\cdots+ \lambda_{n})+ \lambda_{n+1}\\=&a+ \lambda_{n+1}=1.
%\end{split}
%\end{equation*} 
%Siis konveksi verho $C(A)$ sisältää kaikki lineaarikombinaatiot joukon $A$ vektoreista.
%%Tällöin koska $f$ kuuluu konveksiin verhoon $C(hX)$, niin $f$ on lineaarikombinaatio vektorialiavaruuden $hX$ vektoreista. Tällöin olkoon $a_0,a_1,\cdots, a_k \in A$, $\lambda_0,\lambda_1,\cdots, \lambda_k \in \R$ ja $\lambda_i\geq 0$, kaikilla $ i\in\N$ niin, että $$\sum_{i=0}^n \in A \lambda_i f_{a_i}\qquad \text{ missä } \sum_{i=0}^n \lambda_i =1.$$
%%Tällöin $C(A)=\bigcap_{j\in J} A_j$ on pienin konveksi joukko, jolla $A\subset C(A)$, jolloin $C(A)$ on joukon $A$ konveksi verho.\\
%%convex hull = kaikkien A:n sisältävien konveksien joukkojen leikkaus -> konveksi ja sisältää A:n.


%%\begin{maar}

%%\end{maar}
%\end{proof}

\chapter{Banachin avaruus}\label{Banach}

\begin{maar}
Kompakti joukko
\end{maar}

\begin{maar}\emph{Cauchyn jono} 
Jono $(x_n\colon n\in \N$

\end{maar}

\begin{maar}Banachin avaruus on täydellinen normiavaruus.
\end{maar}

\begin{maar}\emph{Separoituva avaruus.} Olkoon $X$ metrinen avaruus ja $\mathcal{B}\subset \mathcal{P}( X)$ epätyhjien avointen osajoukkojen perhe. Tällöin avaruus $X$ on separoituva, jos seuraava ehto pätee: On olemassa numeroituva tiheä osajoukko $\{a_0,a_1,\dots\}=A\subset X$, toisin sanoen, jokaista $B\in\mathcal{B}$ kohti löytyy ainakin yksi $a_i\in A$, jolla $a_i\in B$ jollain $i\in \N$.
\end{maar}

\begin{maar}
\emph{Homotopia}\\
Kaksi kuvausta ovat homotooppisia keskenään, jos ne voidaan muuntaa jatkuvalla kuvauksella toisikseen. Olkoon $f,g\colon X\rightarrow Y$. Sanomme kuvauksen $f$ olevan homotooppinen kuvauksen $g$ kanssa, jos on olemassa jatkuva $h\colon X \times I \rightarrow Y$, jolla $h(x,0)=f(x)$ ja $h(x,1)=g(x)$ kaikilla $x\in X$. Merkitsemme tällöin $f\simeq g$ ja $h\colon f\simeq g$. Kuvaus $h$ on homotopia, joka yhdistää kuvaukset $f$ ja $g$.
\end{maar}


\chapter{Kuratowskin upotuslause}\label{kuratowski}

\begin{lause}
Jokaista metristä avaruutta $(X,d)$ kohti on olemassa normiavaruus $Z$ ja upotus $h\colon X\rightarrow Z$ missä $ hX\subset Z$ on suljettu konveksissa verhossa $C(hX)$. 
\end{lause} 
 
%\vspace{0.5cm}

%\noindent\emph{Todistus.} 
\begin{proof} Olkoon $(X,d)$ metrinen avaruus. Tällöin 
$$d'(x,y)=\dfrac{d(x,y)}{1+d(x,y)} \, , \qquad x,y\in X$$
on lemman \ref{rajmetri} mukaan metriikka joukossa $X$ ja joukon $X$ pisteiden $d'$-etäisyydet toisistaan ovat korkeintaan $1$.

%Olkoon $D$ epätyhjä joukko ja $Raj(D,\R )$ kaikkien avaruudessa $D$ määriteltyjen rajoitettujen funktioiden $f\colon D\rightarrow \R$ vektoriavaruus. Yhtälö $||f||=\sup\{|f(x)|\colon x\in D\}$ määrittelee normin avaruudessa $Raj(D,\R )$. 
%Tätä normia sanotaan avaruuden $Raj(D,\R )$ sup-normiksi.

Olkoon $Z=Raj(X,\R )$ joukon $X$ kaikkien rajoitettujen %jatkuvien 
reaaliarvoisten funktioiden vektoriavaruus. Tällöin avaruudella $Z$ on määritelmän \ref{sup-normi} mukaan muotoa
$$||f_1||=\sup_{x\in X} |f_1(x)| \, , \qquad f_1\in Z$$
oleva sup-normi.
%ja
%$$d(f_1,f_2)=|f_1-f_2| = \sup_{x\in X} |f_1 (x)-f_2 (x)| \, , \qquad f_1, f_2 \in Z .$$

Seuraavaksi määritellään homeomorfismi $h\colon X\rightarrow hX\subset Z$. Tätä varten asetetaan funktio $f_x\in Z$ jokaiselle $x\in X$ yhtälön $f_x (y)=d'(x,y)$ mukaisesti, eli
%Asetamme 
$$h(x)=f_x\, ,\qquad x\in X. $$  Tällöin pätee 
%$$d(f_{x_1} ,f_{x_2} ) \geq |d(x_1 , x_2 )-d(x_2 , x_2 )|=d(x_1 ,x_2 ).$$ 
$$||f_{x_1} - f_{x_2} ||=\sup_{x\in X} |d'(x_1 , x )-d'(x_2 , x )| \geq |d'(x_1 , x_2 )-d'(x_2 , x_2 )|=d'(x_1 ,x_2 ).$$ 
Toisaalta mielivaltaiselle $y\in X$ pätee
$$|f_{x_1} (y)- f_{x_2} (y)|=|d'(x_1 ,y )-d'(x_2 ,y)|\leq d'(x_1 , x_2 ),$$
jolloin siis $$||f_{x_1}-f_{x_2}||=\sup_{x\in X} |d'(x_1 , x )-d'(x_2 , x )|\leq d'(x_1 , x_2 ).$$ Edeltävistä epäyhtälöistä saadaan $||f_{x_1}-f_{x_2}||= d'(x_1 , x_2 )$, jolloin kuvaus $h\colon X\rightarrow hX$  on isometria.
Funktio $h$ on bijektio, sillä kaikilla $x,x'\in X$, $x\neq x'$ pätee $$ f_x (x)=d'(x,x)=0\neq d'(x',x)=f_x' (x)$$ ja edelleen 
$$h(x)=f_x \neq f_x'=h(x') .$$ 
Lemman \ref{isolemma} nojalla funktio $h$ on homeomorfismi kuvajoukolle $hX$.
%\newpage

Seuraavaksi osoitetaan, että $hX$ on suljettu konveksissa verhossa $C(hX)$. 
Tavoitteena on osoittaa, että avaruuden $hX$ alkioista muodostuvan suppenevan jonon raja-arvo kuuluu avaruuteen $ hX$. 
Olkoon $f\in C(hX)$, $f_{x_i}\in hX$ ja $ x_i\in X$ kaikilla $ i\in \N$. 
Oletetaan, että 
$$f=\lim_{n\rightarrow\infty} f_{x_n}.$$ 
%$$f=\lim_{n\rightarrow\infty} f_{x_n}  , \qquad f_{x_i}\in hX \text{ kaikilla }x_{i}\in X, i\in \N.$$ 
Tällöin koska $f$ kuuluu konveksiin verhoon $C(hX)$, niin lemman \ref{konveksilineaari} nojalla $f$ on lineaarikombinaatio vektorialiavaruuden $hX$ vektoreista. Tällöin olkoon $a_0,a_1,\dots, a_k \in X$, $\lambda_0,\lambda_1,\dots, \lambda_k \in \R$ ja $\lambda_i\geq 0$, kaikilla $ i\in\N$ niin, että 
$$f=\sum_{i=0}^k \lambda_i f_{a_i},\qquad \text{ missä } \sum_{i=0}^k \lambda_i =1.$$
Nyt jollain $i\leq k$ pätee $\lambda_i \geq 1/(k+1)$. Tämä seuraa siitä, että jos $\lambda_i<1/(k+1)$ kaikilla $i\leq k$, niin
$$\sum_{i=0}^k \lambda_i < \sum_{i=0}^k \dfrac{1}{k+1}=\dfrac{k+1}{k+1}=1,$$
joka on ristiriidassa oletuksen $\sum_{i=0}^k \lambda_i =1$ kanssa.
%Tällöin on ainakin yksi $i\leq k$, jolla $\lambda_i \geq 1/(k+1)$. Tämä seuraa siitä, että 
%$$\sum_{i=0}^k \frac{1}{k+1}=k\cdot \dfrac{1}{k+1}=\dfrac{k}{k+1}<1 .$$ 
%Tällöin voidaan vaihtaa $\lambda_0$ ja $\lambda_i$ keskenään ja vastaavasti $a_0$ ja $a_i$ keskenään, jolloin saadaan $\lambda_0\geq 1/(k+1)$. Jos jollain $i\neq j$ ja $i,j\leq k$ %ja $i,j\in \N$ 
%pätee $ a_i=a_j$, niin on mahdollista valita uudet $$a_0,a_1,\dots a_i,\dots, a_{j-1},a_{j+1},\dots a_{k}$$ ja vastaavat $$\lambda_0,\lambda_1,\dots, (\lambda_i+\lambda_j),\dots ,\lambda_{j-1},\lambda_{j+1},\dots, \lambda_{k-1}.$$ 
Voidaan siis valita sellaiset $a_0,a_1,\dots, a_k$, $\lambda_0,\lambda_1,\dots \lambda_k$, että 
$$\lambda_0\geq \dfrac{1}{k+1}\text{ ja }f=\sum_{i=0}^k \lambda_i f_{a_i},\qquad \text{ missä } \sum_{i=0}^k \lambda_i =1 .$$ 
Tällöin 
$$||f-f_{x_n}||\geq |f(x_n)-f_{x_n}(x_n)|=|f(x_n)-d'(x_n,x_n)|=|f(x_n)|$$ 
ja edelleen
$$|f(x_n)|=\sum_{i=0}^k \lambda_i f_{a_i}(x_n)\geq \lambda_0 f_{a_0}(x_n)\geq \dfrac{1}{k+1}d(a_0,x_n).$$
Tällöin yhtälöstä $\lim_{n\rightarrow\infty} f_{x_n}=f$ seuraa, että $\lim_{n\rightarrow\infty }x_n=a_0$. \\
Siis $ f=\lim_{n\rightarrow\infty} f_{x_n}=f_{a_0}\in hX$, joten $hX\subset C(hX)$ on suljettu.

\end{proof}


%
%\chapter{Kuratowskin upotuslause v2}\label{Kuratowski2}
%
%\begin{lause}Jokaista metristä avaruutta $X$ kohti on olemassa upotus Banachin avaruuteen $L^{\infty}(X)$ missä $L^{\infty}(X)$ on avaruuden $X$ kaikkien rajoitettujen funktioiden avaruus varustettuna sup-normilla.
%\end{lause}
%
%\begin{proof}
%Olkoon $x_0\in X$ ja 
%$$ x\mapsto s^x,\quad s^x(a)=d(x,a)-d(a,x_0),x\in X.$$ 
%Tällöin kolmioepäyhtälöllä saadaan 
%$$|s^x(a)|=|d(x,a)-d(a,x_0)|\leq d(x,x_0)$$
%ja 
%$$|s^x(a)-s^y(a)|=|d(x,a)-d(y,a)|\leq d(x,y),$$
%jossa yhtäsuuruus pätee, jos $a=x $ tai $a=y $.
%
%\end{proof} 
%
%
%
%
%
%
%
%\chapter{Fréchetin upotuslause}\label{Frechet}
%
%Jokainen separoituva metrinen avaruus $(X,d)$ on upotettavissa (isometrisesti) Banachin avaruuteen $l^\infty$. 
%\begin{lause}Jokaista separoituvaa metristä avaruutta $(X,d)$ kohti on olemassa etäisyydet säilyttävä upotus Banachin avaruuteen $s\colon X\rightarrow l^\infty$. 
%(Huomautus: Banachin avaruus $l^\infty=L^\infty(\N)$ on rajoitettujen jonojen sup-normilla varustettu avaruus.)
%\end{lause}
%
%%\vspace{0.5cm}
%
%%\emph{Todistus.} 
%\begin{proof}
%Olkoon $(X,d)$ separoituva metrinen avaruus. Tällöin on olemassa separoituvuuden määritelmän mukainen tiheä ja numeroituva osajoukko $\{x_0,x_1\cdots \}\subset X$. Tällöin voidaan määritellään 
%$$x\mapsto s^x,\quad s^x_i=d(x,x_i)-d(x_i,x_0).$$
%Tällöin Kuratowskin upotuslauseen antama laajennus kiinnitetyllä pisteellä $x_0$ tiheästä osajoukosta täydelliseen avaruuteen on
%$$\{x_0,x_1,\cdots\}\hookrightarrow L^\infty (\{x_0,x_1,\cdots\})\simeq l^\infty. $$
%Upotus $x\mapsto s^x$ säilyttää etäisyydet, sillä 
%$$|s^a_i-s^b_i|=|d(a,x_i)-d(b,x_i)|\leq d(a,b)\quad \text{kaikilla } a,b\in X,i\in .$$
%Yhtälön yhtäsuuruus pätee jos $x_i=a$ tai $x_i=b$. 
%
%\end{proof}
%
%




%\chapter{spaceholder}\label{nolabel}


\begin{thebibliography}{9}

\bibitem{Bor}
Karol Borsuk: Theory of retracts, %n.\ painos, 
Państwowe Wydawn. Naukowe, 1967.

\bibitem{Topo1}
Jussi Väisälä: Topologia I, 4.\ korjattu painos, Limes ry, 2007.

\bibitem{Topo2}
Jussi Väisälä: Topologia II, 2.\ korjattu painos, Limes ry, 2005.

\bibitem{Hei}
Juha Heinonen: Geometric embeddings of metric spaces, luentomoniste, Jyväskylän yliopisto, 2003
%Sheldon (not really) Ross: A First Course in Probability, 5th edition, Prentice-Hall, 1998.

%\bibitem{Tuo}
%Pekka (not really) Tuominen: Todennäköisyyslaskenta I, 5.\ painos, Limes ry, 2000.

\end{thebibliography}

\end{document}
